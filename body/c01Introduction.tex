%!TEX root = ../AliStrangeJets.tex

\section{Introduction}%
\label{sec:Introduction}

High-energy heavy-ion (A-A) collisions are expected to create a deconfined system with extreme temperature and density, the Quark-Gluon Plasma (QGP)~\cite{Rafelski:126179, Satz:2000bn, Shuryak:1983ni, Jacak:2012dx, Cleymans:1985wb, Bass:1998vz, BraunMunzinger:2007zz}, in which the degrees of freedom are partonic, rather than hadronic.
While the structure and dynamical behavior of the QGP arise at the microscopic level from the interactions between quarks and gluons that are described by Quantum Chromdynamics (QCD)~\cite{Laermann:2003cv, Gupta:2011wh, Bhattacharya:2014ara}.
The interpretation of the heavy-ion results depends crucially on the understanding of results from small collisions systems such as proton-proton(\pp) or proton-nucleus(p-A).
In \pPb collisions, there are not expected hot matter effects.
So it is important to investigate cold nuclear initial and final state effect to used as baseline for heavy-ion collisions~\cite{Salgado:2011wc, Eskola:2016oht}.
In \pp collisions, there are not any hot and cold nuclear initial- and final-state effects.
So it is constitutes a baseline for the nuclear effects in both A-A and p-A collisions.

why strangeness

why jet

%The measurements at the LHC

In this paper, the production of \kzero, \lmb (\almb), \Xis and \Oms in jets and underlying events in \pp at \thirteen and \pPb at \fivenn is reported.
The results presented in this paper significantly improve the precision, also show the centrality classes dependent and extend to multi-strange particle sector, with respect to our previous measurements in both pp (at a different energy) and p–Pb collisions~\cite{V0injet}. 
The paper is structured as follows.
In Sec.~\ref{sec:Detector}, the ALICE apparatus and the data samples used for the analysis are presented. In Sec.~\ref{sec:Analysis} the procedure adopted for charged jet reconstruction, (mulit-)strange particle reconstruction and particle-jet matching strategy is described. 
The systematic uncertainties associated to the measurement are also studied in Sec.~\ref{sec:Analysis}.
The results of the analysis are presented and discussed in Sec.~\ref{sec:Results}.  Finally the paper is briefly summarised in Sec.~\ref{sec:Summary}.